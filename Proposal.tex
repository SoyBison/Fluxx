\documentclass[10pt,letterpaper,prl]{revtex4}
\usepackage[utf8]{inputenc}
\usepackage{amsmath}
\usepackage{amsfonts}
\usepackage{amssymb}
\usepackage{graphicx}
\usepackage[left=2cm,right=2cm,top=2cm,bottom=2cm]{geometry}
\usepackage{float}


\begin{document}
\title{Fluxx Project Proposal}
\author{Coen D. Needell}
\affiliation{Computational Social Science, University of Chicago}
\date{November 1, 2019}
\begin{abstract}
Implementing a card game in python to be used as an interface for game-theoretic simulation study. The card game, Fluxx, will be implemented as it exists in its fifth print edition. The basic goal is to define all of the cards and classes and a game engine. There are stretch goals to implement a couple of basic strategies and a system for analyzing those strategies. An additional stretch goal is to use external machine learning tools to implement a simple adversarial system. 
\end{abstract}

\maketitle

\section*{Background}



\section*{Motivation}

\section*{Implementation}

\section*{Stretch Goals}



\end{document}